\subsection{¿Qu\'e es un electroestimulador muscular?}

Es un dispositivo electr\'onico/el\'ectrico que env\'ia pulsos el\'ectricos a trav\'es de electrodos colocados en la piel en la zona de un m\'usculo de inter\'es. Estos pulsos imitan el potencial de acci\'on que proviene del sistema nervioso central, provocando la contracci\'on muscular. Se utiliza en el campo de la fisioterapia para la recuperaci\'on muscular debido a lesiones, alivio del dolor, etc.

\subsection{Normas}
Las normas que rigen los Equipos electricos medicos est\'an  detallados en la \textbf{IEC6061}. Las siglas IEC significan \textbf{International Electrotechnical Commission}. La Comisión Electrotécnica Internacional (IEC) es la organización líder mundial que prepara y publica normas internacionales para todas las tecnologías eléctricas, electrónicas y afines.

Estas normas abarcan seguridad, rendimiento y compatibilidad.
Dentro de la parte 1 de la norma (IEC 60601-1) se especifica la seguridad básica y
el rendimiento esencial de todos los equipos electro médicos. Las partes
siguientes son requisitos para productos específicos. Particularmente, los requisitos
de seguridad y redimiendo para estimuladores nerviosos y musculares está la norma
IEC 60601-2-10.

\subsubsection{Requisitos generales de seguridad b\'asica y rendimiento esencial (IEC 60601-1)}

\section*{Requerimientos generales}

En el suministro de energ\'ia en equipos conectados a la red, no debe exceder los siguientes voltajes:
\begin{itemize}
    \item $250V$ para equipos de mano.
    \item $250VDC$, monof\'asico AC o $500V$ polifase AC para equipos y sistemas con una entrada nominal $\leq 4kVA$.
    \item $500V$ para todos los dem\'as equipos y sistemas.
\end{itemize}

\section*{Clasificaci\'on de equipos por clase}

\begin{enumerate}

    \item \textbf{Clase I} 

    Equipos eléctricos en los que la protección contra descargas eléctricas tiene
aislamiento básico e  luye una precaución de seguridad adicional a las partes
accesibles o internas de metal están conectada a tierra
    \item \textbf{Clase II}

        Equipos eléctricos con aislamiento doble o el aislamiento reforzado contra
descargas eléctricas, sin que exista una protección de puesta a tierra o dependencia
de las condiciones de instalación.
\end{enumerate}

\section*{Clasificaci\'on de equipos por tipo}

\begin{enumerate}
    \item \textbf{Tipo B}

        Equipos con alimentación interna que tienen un adecuado grado de
protección contra corrientes de fuga y fiabilidad de la conexión a tierra. No tiene
partes aplicables al paciente.
    \item \textbf{Tipo BF}

        Son equipos tipo B con entradas o partes aplicables al paciente, flotante
eléctricamente.
    \item \textbf{Tipo CF}

        Equipo que proporciona un mayor grado de protección que el equipo Tipo
BF. Particularmente en relación con la corriente de fuga permisible.
\end{enumerate}

Los equipos de Tipo B y Tipo BF permiten corriente de fuga de $0,1$ mA en
condiciones normales y de $0,5$ mA en condiciones de fallo. Los equipos de Tipo CF
los valores permitidos de corriente de fuga es de 0,01 mA en condiciones normales
y de 0,05 mA en condiciones de fallo.

\begin{figure}[H]
    \centering
    \includegraphics[width=1.0\linewidth]{tipos.png}
\end{figure}

\section{Señales de seguridad}

El equipo operado por la red deberá estar marcado con la siguiente información:
\begin{itemize}
    \item Los voltajes de suministro nominales o los rangos de voltaje a los que puede conectarse.
    \item Un rango de voltaje de suministro nominal debe tener un guion (-) entre los voltajes mínimo y máximo.
    \item Cuando se dan múltiples voltajes o rangos de suministro nominales deben estar separados por un sólido (/).
\end{itemize}

\section*{Luces indicadoras y controles}

El equipo debe tener luces indicadoras para informar su estado. En la Tabla 1.6 se muestran los colores de los indicadores con su significado.

\begin{table}[H]
\centering
\caption{Colores de luces indicadoras y significados \cite{32}.}
\begin{tabular}{|l|p{9cm}|}
\hline
\textbf{Color} & \textbf{Significado} \\ \hline
Rojo & Advertencia: se requiere una respuesta inmediata del operador. \\ \hline
Amarillo & Precaución: se requiere una respuesta inmediata del operador o que esté atento a su funcionamiento. \\ \hline
Verde & Listo para su uso. \\ \hline
Algún otro color & Significado diferente al de rojo, amarillo o verde. \\ \hline
\end{tabular}
\end{table}
\subsection*{Protección mecánica del cableado}
Los cables internos y el cableado deben estar protegidos para que no exista
fricción en las esquinas y bordes afilados para evitar daños en los procesos de
ensamblaje, apertura o cierre de las cubiertas. Se verifica mediante prueba manual.
\subsection*{Protección contra superficies, esquinas y bordes}
Deben evitarse o cubrirse esquinas afiladas y bordes del equipo que puedan
causar un riesgo inaceptable. Pr ipalmente, se debe prestar atención a los bordes
del marco.
\subsection*{Requerimientos generales para pruebas en el equipo
}
Para realizar pruebas en el equipo se debe considerar las condiciones de
suministro dependiendo si opera en AC, DC o ambas.
\begin{itemize}
    \item El equipo diseñado para ambas formas de suministro AC y DC, se probarán
en condiciones relacionadas con el voltaje menos favorable y la naturaleza
del suministro.
    \item Si el equipo tiene una fuente de alimentación separada ser realizaran pruebas
con dicha fuente.
\end{itemize}

\section{Requisitos particulares (IEC 60601-2-10)}
Esta norma para la seguridad básica y
funcionamiento esencial de los estimuladores nerviosos
y musculares específica los requisitos para la seguridad y rendimiento para
estimuladores destinados a EMS y TENS. Los requisitos críticos de seguridad son:
los \textbf{límites máximos de la corriente de salida} (rms) son 80 mA en CC, 50 mA a 400
Hz, 80 mA a 1500 Hz y 100 mA arriba 1500 Hz (con una carga resistiva de 500
ohmios). También, la energía máxima del pulso no debe exceder los 300 mJ y el
voltaje de salida máximo no debe exceder los 500V. 

Algunas otras importantes a destacar son:
\begin{itemize}
    \item 
       \textbf{201.12.1.101 * Amplitud de salida:} 
Se deberá disponer de un medio para controlar la salida del ESTIMULADOR de mínimo a máximo de forma continua o en  rementos discretos de no más de 1 mA o 1 V por incremento. En su ajuste mínimo, la salida no deberá superar el 2 \% de la disponible en el ajuste máximo del control.
El cumplimiento se verifica mediante inspección y medición utilizando la impedancia de carga menos favorable dentro del rango de impedancia de carga especificado en los DOCUMENTOS ADJUNTOS.
\item \textbf{201.12.1.102 * Parámetros de PULSO
:} Los valores de DURACIÓN DE PULSO, frecuencias de repetición de PULSO y amplitudes,  luyendo cualquier corriente continua (CC). El componente, ya sea causado por un desplazamiento o por una forma de onda asimétrica, como se describe en los DOCUMENTOS ADJUNTOS o se indica en el EQUIPO ME (véase 201.7.9.2), no deberá desviarse en más de $\pm 20$ \% cuando se mide con una resistencia de carga dentro del rango especificado en los DOCUMENTOS ADJUNTOS (véase 201.7.9.3). 
\end{itemize}    

\section{Estándares en el diseño de dispositivos electrónicos}
Para diseñar, fabricar y producir un producto electrónico confiable y de alta
calidad están los estándares IPC.


Para especificaciones de diseño se usa la familia:
\begin{itemize}
    \item IPC 2220.
    \item IPC 7351.
    \end{itemize}
Para diseños generales está la norma IPC 2221 y en su jerarquía se encuentra
la IPC 2222 que es para PCB rígidos

\section{Diseño de estructura de tablero impreso orgánico
rígido (IPC 2222)}
Esta norma proporciona información sobre los requisitos detallados para el
diseño de tableros impresos rígidos orgánicos y complementa la información
contenida en IPC 2221.

\subsection*{Tipo de placa
}
La norma clasifica en seis tipos:
\begin{itemize}
\item Tipo 1: Placa impresa a simple lado.
\item Tipo 2: Placa impresa a doble lado.
\item Tipo 3: Placa multicapa sin vías ciegas.
\item Tipo 4: Placa multicapa con vías ciegas.
\item Tipo 5: Placa de núcleo de metal multicapa sin vías ciegas.
\item Tipo 6: Placa de núcleo de metal multicapa con vías ciegas.
\end{itemize}

\subsection*{Tipo de ensamblaje
}
Identificar el tipo de ensamble ayuda para describir de mejor manera si los
componentes están montados en uno o los dos lados de la placa. Existen dos tipos:
tipo 1 define componentes a un solo lado y el tipo 2 contiene componentes en ambos
lados. No se recomienda un tipo 2 en una placa clase A.
\subsection*{Material de la placa
}
Las placas deben fabricarse con los materiales especificados en las tablas de la norma y seleccionar el material adecuado acorde a la de la misma norma



